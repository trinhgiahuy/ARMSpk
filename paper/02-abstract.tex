% IPDPS requirement !Abstract (Maximum 250 words)!
\begin{abstract}
Among the (uncontended) common wisdom in High-Performance Computing (HPC) is the applications’ need for large amount of double-precision support in hardware.  Hardware manufacturers, the TOP500 list, and (rarely revisited) legacy software have without doubt followed and contributed to this~view. 

In this paper, we challenge that wisdom, and we do so by exhaustively comparing a large number of HPC proxy application on two processors: Intel’s Knights Landing (KNL) and Knights Mill (KNM). Although similar, the KNM and KNL architecturally deviate at one important point: the silicon area devoted to double-precision arithmetic’s. This fortunate discrepancy allows us to empirically quantify the performance impact in reducing the amount of hardware double-precision arithmetic. %this part reads strange

Our analysis shows that this common wisdom might not always be right. We find that the investigated HPC proxy applications do allow for a (significant) reduction in double-precision with little-to-no performance implications. With the advent of a failing of Moore's law, our results partially reinforce the view taken by modern industry (e.g. upcoming Fujitsu ARM64FX) to integrate hybrid-precision hardware units.

%\cJD{PLACEHOLDER: (Maximum 250 words)!}
%Common perception in supercomputing is that double precision floating point
%calculations is what matters, with respect to both application requirements and
%performance. Accordingly, chip manufacturers have allocated a significant
%portion of chip area to double precision FPUs, in turn reducing the chip's
%compute-per-silicon ratio, and also the bandwidth utilization.
%We conduct an exhaustive FPU-requirement and performance study using 22 HPC
%(proxy/mini) applications from various scientific domains, which comprise the
%majority of CPU cycles in HPC. These applications will be executed on two
%target platforms with overall similar characteristic, however drastically
%different amount of double precision units. Hence, this comparison will give
%vendors and the rest of the community a chance to investigate the precision
%or unit requirements, and identify performance bottlenecks in modern HPC codes,
%to guide the procurement towards more/less FP64, FP32, ..., or faster/bigger
%memory and caches, or more cores instead.
%\cJD{PLACEHOLDER: (Maximum 250 words)!}
\end{abstract}
